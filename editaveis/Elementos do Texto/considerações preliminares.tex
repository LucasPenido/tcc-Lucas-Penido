\chapter[Considerações Preliminares]{Considerações Preliminares}

\section{Conclusão}

Inicialmente foram identificadas algumas bibliotecas em Dart que implementam apenas o algoritmo Ed25519, as duas mais maduras são: a \href{https://pub.dev/packages/cryptography}{Cryptography} e a \href{https://pub.dev/packages/pinenacl}{PineNaCl}. Para fins de estudo e de validação, foram realizadas validações cruzadas entre a biblioteca Cryptography e a \href{https://pynacl.readthedocs.io/en/1.4.0/signing/}{PyNaCl} escrita em Python, onde se assinava uma mensagem em uma biblioteca e a verificava na outra.

Em seguida, iniciou-se um estudo com o certificado digital da AC Raiz da ICP-Brasil v6, no qual é utilizado o algoritmo Ed448. Já que não foi identificada nenhuma biblioteca que realizasse a leitura e verificação desse certificado em Dart,  foi utilizada a biblioteca \href{https://cryptography.io/en/latest/x509/reference/}{Cryptography} em Python para realização do estudo.

Houve uma tentativa de verificação do certificado digital da AC Raiz da ICP-Brasil v7, o qual é utilizado o algoritmo Ed521, porém não foi encontrado nenhuma biblioteca que realizasse sua verificação. No entanto, foi identificado a biblioteca \href{https://github.com/cslashm/ECPy}{ECPy} em Python, que possui a implementação EdDSA e é bastante flexível. Dessa forma, foi acrescentado o algoritmo Ed521 à ela e assim conseguiu-se realizar a verificação desse cerificado digital, além de ser possível gerar chaves públicas e privadas e realizar a assinatura de mensagens. 

A presente proposta de trabalho, buscará a criação de módulos, escritos na linguagem de programação Dart, que implementem os algoritmos Ed448 e Ed521. A relevância dessa implementação se dá pela falta de um componente de software que implemente tais algoritmos nessa linguagem.

\section{Cronograma}

A seguir serão apresentadas as atividades executadas no período de elaboração do TCC 1, além das atividades planejadas para serem concluídas durante o TCC 2.

\begin{enumerate}
    \item Estudo sobre assinaturas digitais;
    \item Estudo sobre a linguagem Dart;
    \item Estudo sobre algoritmos definidos pela ICP-Brasil (Ed25519, Ed448 e Ed521);
    \item Implementação do protótipo em Python do algoritmo Ed521;
    \item Redigir o trabalho referente ao TCC 1;
    \item Apresentação do TCC 1;
    \item Desenvolver os módulos em Dart;
    \item Realizar testes e melhorias no módulo;
    \item Redigir o trabalho referente ao TCC 2;
    \item Apresentação do TCC 2;
\end{enumerate}

Para uma melhor apresentação das atividades e suas datas, as mesmas serão apresentadas em duas tabelas. Na Tabela \ref{cronoTCC1} estão definidas as atividades realizadas durante o desenvolvimento do TCC 1. Na Tabela \ref{cronoTCC2} está estabelecido o planejamento inicial das atividades que serão realizadas durante o TCC 2.

\begin{table}[H]
\caption{Cronograma referente ao TCC 1}
\label{cronoTCC1}
\resizebox{\textwidth}{!}{%
\begin{tabular}{|l|c|c|c|c|}
\hline
 & \multicolumn{1}{l|}{\textbf{Fevereiro}} & \multicolumn{1}{l|}{\textbf{Março}} & \multicolumn{1}{l|}{\textbf{Abril}} & \multicolumn{1}{l|}{\textbf{Maio}} \\ \hline
Estudo sobre assinaturas digitais & X & X &  &  \\ \hline
Estudo sobre a linguagem Dart & X & X &  &  \\ \hline
Estudo sobre algoritmos definidos pela ICP-Brasil (Ed25519, Ed448 e Ed521) &  & X & X &  \\ \hline
Implementação do protótipo em Python do algoritmo Ed521 &  &  & X &  \\ \hline
Redigir o trabalho referente ao TCC 1 &  & X & X &  \\ \hline
Apresentação do TCC 1 &  &  &  & X \\ \hline
\end{tabular}%
}
\end{table}

\begin{table}[H]
\caption{Cronograma referente ao TCC 2}
\label{cronoTCC2}
\resizebox{\textwidth}{!}{%
\begin{tabular}{|l|c|c|c|c|c|}
\hline
 & \multicolumn{1}{l|}{\textbf{Julho}} & \multicolumn{1}{l|}{\textbf{Agosto}} & \multicolumn{1}{l|}{\textbf{Setembro}} & \multicolumn{1}{l|}{\textbf{Outubro}} & \multicolumn{1}{l|}{\textbf{Novembro}} \\ \hline
Desenvolver os módulos em Dart & X & X & X &  &  \\ \hline
Realizar testes e melhorias no módulo &  & X & X & X &  \\ \hline
Redigir o trabalho referente ao TCC 2 &  & X & X & X &  \\ \hline
Apresentação do TCC 2 &  &  &  &  & X \\ \hline
\end{tabular}%
}
\end{table}