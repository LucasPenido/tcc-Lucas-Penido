\chapter[Introdução]{Introdução}

\section{Justificativa}

A assinatura digital é o desenvolvimento mais importante a partir dos trabalhos sobre criptografia de chave pública, sendo utilizada como garantia de proteção em transações eletrônicas dentre outros serviços na Web. A assinatura digital trouxe uma série de benefícios, como a criação de Certificados Digitais, assegurando e facilitando a identificação de pessoas físicas, jurídicas, ou até mesmo de máquinas na internet. 

No âmbito das assinaturas digitais, o Algoritmo de Assinatura Digital da Curva Elíptica de Edwards (EdDSA) tem se tornado cada vez mais utilizado, uma vez que tem se demonstrado mais simples, mais seguro e mais rápido do que outros algoritmos, como o Algoritmo de Assinatura Digital de Curvas Elípticas (ECDSA), além de depender da dificuldade de se calcular logaritmos discretos em curvas elípticas. 

ICP-Brasil optou por recomendar a utilização de EdDSA na sua busca pela implementação de algoritmos e suítes de assinatura digitais, com a finalidade de desburocratizar e agilizar, com a segurança técnica e jurídica necessária, todos os procedimentos e segmentos da sociedade brasileira, diminuindo os custos e fraudes, tornando a vida do cidadão melhor \cite{iti2018}.

Este trabalho tem como proposta implementar os algoritmos de assinatura definidos pelo ICP-Brasil na linguagem de programação Dart. Essa linguagem foi recentemente desenvolvida pela Google e vem tomando bastante espaço no mundo da tecnologia por conta da tecnologia Flutter. 

\section{Problema de Pesquisa}

\section{Objetivos}

\subsection{Objetivo Geral}

\subsection{Objetivos Específicos}

