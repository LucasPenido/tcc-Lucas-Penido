\chapter[Introdução]{Introdução}
% \addcontentsline{toc}{chapter}{Introdução}

% Este documento apresenta considerações gerais e preliminares relacionadas 
% à redação de relatórios de Projeto de Graduação da Faculdade UnB Gama 
% (FGA). São abordados os diferentes aspectos sobre a estrutura do trabalho, 
% uso de programas de auxilio a edição, tiragem de cópias, encadernação, etc.

% Este template é uma adaptação do ABNTeX2\footnote{\url{https://github.com/abntex/abntex2/}}.

\chapter[Justificativa]{Justificativa}

A assinatura digital é o desenvolvimento mais importante a partir dos trabalhos sobre criptografia de chave pública, sendo utilizada como garantia de proteção em transações eletrônicas e dentre outros serviços na Web. A assinatura digital trouxe uma série de benefícios, como a criação de Certificados Digitais, assegurando e facilitando a identificação de pessoas físicas, jurídicas, ou até mesmo de máquinas na internet. 

No âmbito das assinaturas digitais, o Algoritmo de Assinatura Digital da Curva de Edwards (EdDSA) tem se tornado cada vez mais utilizado, uma vez que tem se demonstrado mais simples, mais seguro e mais rápido do que outros algoritmos, como o Algoritmo de Assinatura Digital de Curvas Elípticas (ECDSA), além de depender da dificuldade de se calcular logaritmos discretos em curvas elípticas. 

Desta forma a ICP-Brasil busca a implementação de algoritmos e suítes de assinatura digitais, com a finalidade de desburocratizar e agilizar, com a segurança técnica e jurídica necessária, todos os procedimentos e segmentos da sociedade brasileira, diminuindo os custos e fraudes, tornando a vida do cidadão melhor \cite{iti2018}.

Este trabalho tem como proposta implementar os algoritmos de assinatura definidos pelo ICP-Brasil na linguagem de programação Dart. Essa linguagem foi recentemente desenvolvida pela Google e vem tomando bastante espaço no mundo da tecnologia por conta da tecnologia Flutter. 

\chapter[Problema de Pesquisa]{Problema de Pesquisa}

\chapter[Objetivos]{Objetivos}

\section{Objetivo Geral}

\section{Objetivos Específicos}

